% Document class and font size
\documentclass[a4paper,9pt]{extarticle}

% Packages
\usepackage{CJKutf8}
\usepackage[utf8]{inputenc} % For input encoding
\usepackage{geometry} % For page margins
\geometry{letterpaper, margin=0.75in} % Set paper size and margins
\usepackage{titlesec} % For section title formatting
\usepackage{enumitem} % For itemized list formatting
\usepackage{tabularx}
\usepackage{fancyhdr}
\usepackage[hidelinks]{hyperref}
\hypersetup{breaklinks=true}
\hfuzz=20pt


% Formatting
\setlist{noitemsep} % Removes item separation
\titleformat{\section}{\large\bfseries}{\thesection}{1em}{}[\titlerule] % Section title format
\titlespacing*{\section}{0pt}{\baselineskip}{\baselineskip} % Section title spacing
%%%%%%%%%%

% Begin document
\begin{document}

% Disable page numbers
\pagestyle{fancy}
\renewcommand{\headrulewidth}{0pt}
\fancyhead{}
\fancyhead[L]{\textit{Jongpil Jeong}}
\fancyhead[R]{\textit{November 2025}}
\thispagestyle{empty} % Remove header from the first page

% Header
\begin{flushleft}
\href{https://scholar.google.com/citations?user=O-3pYeQAAAAJ&hl=en}{\textbf{\LARGE Jongpil Jeong}}\\[2pt] % Name
Master course \\ 
Department of Creative Informatics \\ 
Graduate School of Computer Science and Systems Engineering \\ 
Kyushu Institute of Technology \\ 
680--4, Kawazu, Iizuka, Fukuoka, Japan \\ 
\href{mailto:jeong.jongpil383@mail.kyutech.jp}{jeong.jongpil383@mail.kyutech.jp} \\%| \href{https://www.linkedin.com/in/JohnDoe}{www.linkedin.com/in/JohnDoe} | \href{https://JohnDoe/en/staff/single/59}{Organization page} \\ % Contact info \\
\href{tel:+82-10-8912-3304}{+82-10-8912-3304} [KR] \\\href{tel:+81-90-7269-3467}{+81-90-7269-3467} [JP] 
\end{flushleft} 


\section*{EDUCATION}
\textbf{Kyushu Institute of Technology} \hfill Iizuka, Fukuoka, Japan\\
Master of Engineering, Graduate School of Computer Science and Systems Engineering \hfill \textit{Apr. 2024 — Mar. 2026(Expected)} \\
Department of Creative Informatics (Computer Science and Networks)  \hfill Cumulative GPA:\textbf{3.02/4.00} \\
Thesis (in progress): \textit{“Visibility restoration via spatial frequency domain interpretation under harsh conditions”} \\
Advisor: \href{https://scholar.google.com/citations?user=9sv1-5QAAAAJ&hl=en&oi=ao}{Prof. Min-Chul Lee} \\

\noindent
\textbf{Dong-A University} \hfill Busan, Korea \\
Bachelor of Engineering in Electronics Engineering \hfill \textit{Mar. 2018 — Feb. 2024} \\
(Top 10\%; 156 credits completed / 150 required) \hfill Cumulative GPA:\textbf{3.91/4.50} \\
Recipient of Academic Excellence Scholarships (5 semesters) \\

% Education Section
\section*{RESEARCH INTERESTS}

\begin{itemize}
    % \item Three-dimensional imaging system
    \item \textbf{Image Processing} / Computer Vision
    \item \textbf{Biomedical Imaging System}
    \item \textbf{Digital Holographic Microscopy (DHM)}
    % \item Computer vision
    \item Optical Signal Processing
    \item \textbf{Denoising / Deblurring Algorithm}
    % \item Data science
    \item Deep Learning / Machine Learning. \\
\end{itemize} 

\section*{SKILLS}
\begin{tabularx}{\textwidth}{@{}lX@{}}
\textbf{Languages}     & Korean (Native), English (\href{https://github.com/user-attachments/assets/f578b432-f7a8-410e-851c-852e9ae55378}{OPIc IH}), Japanese (Intermediate) \\ 
\textbf{Programming}   & C/C++, Python, MATLAB \\
\textbf{Libraries}     & PyTorch, TensorFlow, OpenCV, Qt, Pandas, NumPy, SciPy, Plotly \\
\textbf{Tools / OS}    & \LaTeX, Digit Cam, Spinnaker SDK, Pylon SDK, Git, Blender \\
\textbf{Affiliatios}    & IEEE Student Member, SPIE Student Member \\ \\
\end{tabularx}

% % Professional Experience
% \section*{Professional Experience}
% % HITACHI
% \noindent 
% \textbf{Development of pre-hatching sex determination technology for chickens} \hfill Fukuoka, Tokyo, Japan, Netherlands % Project name and location
% \begin{itemize}
%     \item \textit{Research Assistant} \hfill Apr. 2026 | Nov. 2026
%     \begin{itemize}
%         \item Developed the core algorithm for high accuracy and efficiency.
%         \item Led the end-to-end development of core algorithm, achieving improvement in accuracy from 56\% to 87\% over previous methods.
%         \item Designed and implemented a Graphic User Interface (GUI) to streamline user interaction.
%         \item Created and packaged the necessary installer for seamless program distribution and development.
%     \end{itemize}
% \item Grant type: Joint Research (Funded by a private industry partner under NDA) \\
% % \hfill (Participation period) % Project link and duration
% \end{itemize}

% Projects Section
\section*{RESEARCH PROJECTS}
% 소방청
\noindent 
\textbf{Development of AR rescue support system for disaster sites with poor visibility} \hfill Fukuoka, Tokyo, Japan\\% Project name and location
\textbf{using optical signal processing and deep learning} \hfill May. 2024 | Mar. 2025 (Expected)
\begin{itemize}
    \item \textit{Post-Master's Researcher}  
    \begin{itemize}
        \item Designing and optimizing algorithms based on the ARMS to effectively remove non-uniform scattering media.
        \item Currently working on embedding and optimizing the algorithm for real-time processing on the Microprocessor.
    \end{itemize}
    \item Grant number: JPJ000255 (Funded by Fire and Disaster Management Agency of Japan) 
    \item Grant type: Consigned research  %(Funded by Fire and Disaster Management Agency of Japan)\\% Project link and duration
\end{itemize}

% KAKEN
\noindent 
\textbf{Image processing techniques for visualizing poor-visibility scenes under scattering media} \hfill Fukuoka, Japan % Project name and location
\begin{itemize}
    \item \textit{Research Assistant} \hfill Apr. 2024 | Mar.2027 (Expected)  
    \begin{itemize}
        \item Led the development of the Adaptive Removal via Mask for Scatter (ARMS) algorithm to enhance visibility in images degraded by scattering media.
        \item Resulted in multiple patent applications, including international patents (PCT), protecting the core technology and methodology.
    \end{itemize}
    \item Grant number: 24K01120 (Funded by KAKEN, JSPS) %\hfill (Project: Apr. 2024--Mar. 2027)  % Project link and duration
\end{itemize}


% KAKEN
\noindent
\textbf{Development of a 360-degree Digital Holographic Microscope (DHM) for 3D object profiling} \hfill Fukuoka, Japan % Project name and location
\begin{itemize}
    \item \textit{Research Assistant} \hfill Apr. 2024 | Mar. 2025 
    \begin{itemize}
        \item Statistical approach implemented to mitigate phase error and denoise 3D profile reconstructions.
        \item Applied a Kalman filter to suppress noise originating from the DC term in the Fourier domain.
    \end{itemize}
    \item Grant number: 23K19964 (Funded by KAKEN, JSPS)    % Project link and duration 
\end{itemize}

% HITACHI
\noindent 
\textbf{Development of pre-hatching sex determination technology for chickens} \hfill Fukuoka, Gifu, Niigata, Ibaraki, Japan % Project name and location
\begin{itemize}
    \item \textit{Research Assistant} \hfill Apr. 2024 | Nov. 2024
    \begin{itemize}
        \item Developed the core algorithm for high accuracy and efficiency.
        \item Led the end-to-end development of core algorithm, achieving improvement in accuracy from 56\% to 87\% over previous methods.
        \item Designed and implemented a Graphic User Interface (GUI) to streamline user interaction.
        \item Created and packaged the necessary installer for seamless program distribution and development.
    \end{itemize}
\item Grant type: Joint Research (Funded by a private industry partner under NDA) \\
% \hfill (Participation period) % Project link and duration
\end{itemize}

% \newpage
% Paper publications
\section*{PUBLICATIONS}

% \noindent
\subsection*{Journal}
\begin{enumerate}[label={[\arabic*]}, start=1]
    % % MDPI
    % \item 
    % \underbar{\textbf{J. Jeong}}, and M.-C. Lee,
    % ``Scattering Medium Removal Using Adaptive Masks for Scatter in the Spatial Frequency Domain,''
    % \textit{IEEE Access},
    % vol. 13, pp. 72769--72777, 
    % 2025. \\
    % DOI:\href{https://doi.org/10.1109/ACCESS.2025.3563369}{10.1109/ACCESS.2025.3563369} 

    % IEEE Access
    \item 
    \underbar{\textbf{J. Jeong}}, and M.-C. Lee,
    ``Scattering Medium Removal Using Adaptive Masks for Scatter in the Spatial Frequency Domain,''
    \textit{IEEE Access},
    vol. 13, pp. 72769--72777, 
    2025. \\
    DOI:\href{https://doi.org/10.1109/ACCESS.2025.3563369}{10.1109/ACCESS.2025.3563369} 
\end{enumerate} 

\subsection*{Conference}
\begin{enumerate}[label={[\arabic*]}, start=1]
    
    % 2025 ICCAS
    %--------------------------------------------------------------------------------------------------------------%
    \item 
    Y. Takahashi, \underbar{\textbf{J. Jeong}}, M. Cho, and M.--C. Lee, 
    ``A research on scattering media removal and photon estimation using COLaNoPS,''
    \textit{Proc. ICCAS 2025}, (IEEE),
    Incheon, Korea. 
    % DOI: \href{https://doi.org/}{xxxxxxxxxxxxxxxxxxxxx},
    (Accepted)

    % 2025 ICMV
    %--------------------------------------------------------------------------------------------------------------%
    \underbar{\textbf{J. Jeong}}, M. Cho, and M.--C. Lee, 
    \item 
    ``Advanced scattering media removal by modified ARMS and restoration of color information,'' 
    \textit{Proc. ICMV 2025}, (SPIE),
    Paris, France. 
    % DOI: \href{https://doi.org/}{xxxxxxxxxxxxxxxxxxxxx},
    (Accepted). \\
    - \href{https://github.com/user-attachments/assets/53461ed5-0c97-4db9-8181-8551847832b3}{\textbf{Best Poster Presentation Award}}

    \item 
    S. Song, \underbar{\textbf{J. Jeong}}, M. Cho, and M.--C. Lee, 
    ``Single Haze Removal Method using Peplography,'' 
    \textit{Proc. ICMV 2025}, (SPIE),
    Paris, France. 
    % DOI: \href{https://doi.org/}{xxxxxxxxxxxxxxxxxxxxx},
    (Accepted).

    % 2025 ITC-CSCC
    %--------------------------------------------------------------------------------------------------------------%
    \item 
    \underbar{\textbf{J. Jeong}}, M. Cho, and M.--C. Lee, 
    ``Scattering media removal under the harsh conditions using adaptive removal via mask for scatter,'' 
    \textit{Proc. ITC-CSCC 2025}, (IEEE),
    Seoul, Korea. \\
    DOI:\href{https://doi.org/10.1109/ITC-CSCC66376.2025.11137793}{10.1109/ITC-CSCC66376.2025.11137793}
    
    \item 
    K. Nakamura, \underbar{\textbf{J. Jeong}}, M. Cho, and M.--C. Lee, 
    ``Adaptive Optimization of Kalman Filtering in Digital Holographic Microscopy for Improved Noise Reduction,'' 
    \textit{Proc. ITC-CSCC 2025}, (IEEE),
    Seoul, Korea. \\
    DOI:\href{https://doi.org/10.1109/ITC-CSCC66376.2025.11137616}{10.1109/ITC-CSCC66376.2025.11137616}
    
    \item 
    S. Kim, \underbar{\textbf{J. Jeong}}, M. Cho, and M.--C. Lee, 
    ``Advanced double random phase encryption for simultaneous two primary data,'' 
    \textit{Proc. ITC-CSCC 2025}, (IEEE),
    Seoul, Korea. \\
    DOI:\href{https://doi.org/10.1109/ITC-CSCC66376.2025.11137702}{10.1109/ITC-CSCC66376.2025.11137702}
    
    % 2024 ICCAS
    %--------------------------------------------------------------------------------------------------------------%
    \item 
    T. Ono, \underbar{\textbf{J. Jeong}}, H.--W. Kim, M. Cho, and M.--C. Lee, 
    ``Kalman filtering optimization in digital holographic microscopy (DHM),'' 
    \textit{Proc. ICCAS 2024}, (IEEE),
    Jeju, Korea.\\
    DOI:\href{https://doi.org/10.23919/ICCAS63016.2024.10773243}{10.23919/ICCAS63016.2024.10773243} 
    
    
    % 2024 JCSSE
    %--------------------------------------------------------------------------------------------------------------%
    \item 
    \underbar{\textbf{J. Jeong}}, H.--W. Kim, M. Cho, and M.--C. Lee, 
    ``A study of noise reduction algorithm using statistical optimization in digital holographic microscopy,'' 
    \textit{Proc. JCSSE 2024}, (IEEE),
    Phuket, Thailand.\\
    DOI:\href{https://doi.org/10.1109/JCSSE61278.2024.10613728}{10.1109/JCSSE61278.2024.10613728} 
\end{enumerate} 

\subsection*{Patents}
\subsubsection*{International Patents (PCT)}
\begin{enumerate}[label={[\arabic*]}, start=1]
    \item 
    M.--C. Lee and \underbar{\textbf{J. Jeong}},  \hfill {Pending} \\%\hfill {Oct. 27, 2025}\\
    ``Image processing apparatus, image processing method, and image processing program,''                         \\
    PCT Patent Reference No. PCT/JP2025/037681
\end{enumerate}
\subsubsection*{Domestic Patents \hfill \normalfont{Japan}} 
\normalfont{
    \begin{enumerate}[label={[\arabic*]}, start=1]
        \item 
        M. Kamide, O. Shiba, K. Ozawa, T. Nakaya, M.--C. Lee, \underbar{\textbf{J. Jeong}}, and T. Tagami,           \hfill {Sep. 18, 2025} \\
        ``Sex identification service provision system and sex identification service provision method,''    \\
        Japanese Patent Application No. 2025--154965.
        \item 
        M.--C. Lee and \underbar{\textbf{J. Jeong}},                                                                \hfill {Jun. 10, 2025} \\
        ``Image processing apparatus, image processing method, and image processing program,'' \\
        Japanese Patent Application No. 2025--097331. 
        % (Not publicly accessible at this time due to confidentiality under Japanese patent law.)
        \item 
        M.--C. Lee and \underbar{\textbf{J. Jeong}},                                                                \hfill {Aug. 30, 2024} \\
        ``Image processing apparatus, image processing method, and image processing program,'' \\
        Japanese Patent Application No. 2024--214715.
        
        \item[*] In accordance with Japanese patent law, these applications are kept confidential and are not publicly disclosed for 18 months following their filing. 
        \href{https://hyokadb02.jimu.kyutech.ac.jp/html/215_en.html}{(Reference)}
    
    \end{enumerate} 
}

% \newpage% Experience Section
\section*{Additional Research Experience}

\noindent
\textbf{\href{https://3cholab.wordpress.com/정보/}{Computational, Holographic and Optical signal processing Lab.}} \hfill Anseong, Gyunggi-do, Korea\\ % Company name and location
\textbf{at Hankyung National University} \hfill Jan. 2024 | Feb. 2024 \\
\textit{Visiting Research Student} \\
Advisor: Prof. \href{https://orcid.org/0000-0003-2896-770X}{Myugnjin Cho}
\begin{itemize}
    \item Integral Imaging Systems
    \item Principle of image encryption such as double random phase encryption (DRPE) \\
\end{itemize}

\noindent
\textbf{\href{https://leelab.csn.kyutech.ac.jp}{3D Optical Imaging System Lab.} at Kyushu Institute of Technology} \hfill Fukuoka, Japan\\ 
\textit{Short-Term Visiting Researcher} (Winter \& Summer ) \hfill Jan. |  Feb. \& Jul.  | Aug. 2023\\
Advisor: Prof. \href{https://orcid.org/0000-0001-8469-0288}{Min-Chul Lee}
\begin{itemize}
    \item Studied Digital Holographic Microscopy(DHM) and phase error correction  
    \item Developed noise reduction algorithms under low-light (photon-starved) conditions
    \item Restored low-light images using photon-counting techniques 
    \item Visibility enhancement under harsh conditions through the scattering media \\
\end{itemize}

\noindent
\textbf{\href{https://soc.donga.ac.kr/soc/Main.do}{SoC Design Lab.} at Dong-A University} \hfill Busan, Korea\\ % Company name and location
\textit{Undergraduate Research Intern} \hfill Sep. 2022 | Jul. 2023 \\
Advisor: Prof. \href{https://orcid.org/0000-0001-6716-5799}{Bongsoon Kang}
\begin{itemize}
\item Completed the \href{https://github.com/user-attachments/assets/583762df-f208-4216-b836-a6a1921ec34b}{IDEC SoC Design Course} (48 hours, Spring 2023), which initiated my interest in image processing and computational systems. 
\item Topics covered: Verilog HDL fundamentals, structural and dataflow modeling, and algorithmic-level design.
\begin{itemize}
    \item Basic image processing techniques.
    \item Principle of machine learning.
    \item Programming with C/C++, MATLAB, Python, and Verilog \\
\end{itemize}
\end{itemize}



\section*{Awards, Scholarships and Tuition Waivers}
\large{\textbf{\href{https://icmv.org}{2025 18th International Conference on Machine Vision}}} \hfill Paris, France
\begin{itemize}
    \item \href{https://github.com/user-attachments/assets/53461ed5-0c97-4db9-8181-8551847832b3}{Best Poster Presentation Award}
\end{itemize}
\noindent
% Company name and location
\large{\textbf{Kyushu Institute of Technology}} \hfill Fukuoka, Japan
% \subsection*{Kyushu Institute of Technology} \hfill Fukuoka, Japan\\
% \subsubsection*{Waivers from Tuition Fees}
\begin{itemize}
    \item \textbf{Waivers from Tuition Fees}
    \begin{itemize}
        \item 2025, $1^{st}$ semester 
        \item 2024, $1^{st} / 2^{nd}$ semester 
    \end{itemize} 
\end{itemize}

\large{\textbf{Dong-A University}} \hfill Busan, Korea
% \subsubsection*{Academic Excellence Scholarship}
\begin{itemize}
    \item \textbf{Academic Excellence Scholarship}
    \begin{itemize}
        \item 2023, $1^{st} / 2^{nd}$ semester 
        \item 2022, $1^{st}$ semester 
        \item 2021, $2^{nd}$ semester 
        \item 2018, $2^{nd}$ semester 
    \end{itemize} 
    % \subsubsection*{Advisory Professor Scholarship from Dong-A University}
    \item \textbf{Advisory Professor Scholarship from Dong-A University}
    \begin{itemize}
        \item 2022, $2^{nd}$ semester 
    \end{itemize} 
    % \subsubsection*{Undergraduate Education Assistant Scholarship from Dong-A University}
    \item \textbf{Undergraduate Education Assistant Scholarship from Dong-A University}
    \begin{itemize}
        \item 2023, $1^{st}/2^{nd}$ semester 
        \item 2022, $2^{nd}$ semester \\
    \end{itemize} 
\end{itemize}


% % \newpage
% \section*{Relevant Coursework}
% % KOCW
% \paragraph{\large{\href{https://www.kocw.net/home/index.do}{Korea OpenCourseWare (KOCW)}}} \hfill Online, Korea
% \begin{itemize}
%     \item \href{https://github.com/user-attachments/assets/d8024d63-359a-493b-9cb2-774e36a03f62}{Digital Image Processing}                                              \hfill 2025
% \end{itemize}

% % IDEC
% \paragraph{\large{\href{https://www.idec.or.kr/main/}{IC Design Education Center (IDEC)}}}  \hfill Online, Korea
% \begin{itemize}
%     %2024
%     \item \href{https://github.com/user-attachments/assets/234ad610-a051-4140-9f20-48ca4d8e4d42}{Implementation of CNN's FPGA with Verilog HDL}                         \hfill 2024
%     \item \href{https://github.com/user-attachments/assets/68b0ae05-462b-4750-93ba-1bc4072dab66}{Design embedded systems based on FPGA}
%     \item \href{https://github.com/user-attachments/assets/8fbda935-4b41-44f7-b83a-8d284f003168}{Data structure and algorithm}                                          
%     \item \href{https://github.com/user-attachments/assets/805bbc13-3e6e-43d6-af75-df31f728d607}{FreeRTOS porting and utilization through Cortex-M processor}           
%     \item \href{https://github.com/user-attachments/assets/707a4f8e-3b5a-4246-a763-81e2729f9cb2}{MIMO - theory and improvement}                                         
% \end{itemize}

% \begin{itemize}
%     %2023
%     % \vspace{0.5em}
%     \item \href{https://github.com/user-attachments/assets/2d45999d-1f66-467c-b956-e16b5ed71593}{Stereovision for autonomous driving system}                            \hfill 2023
%     \item \href{https://github.com/user-attachments/assets/32bfc1d6-18cf-44a4-83dc-209b4f4686ac}{Design digital system utilized Verilog}
%     \item \href{https://github.com/user-attachments/assets/094a58df-f320-45fb-aba3-707f4a7fd8f7}{Neural network hardware accelerator ``Architecture''}
%     \item \href{https://github.com/user-attachments/assets/9f74252d-eae5-40b2-b305-ea90576100aa}{DSP with MATLAB}
%     \item \href{https://github.com/user-attachments/assets/0fc8c167-bef1-4857-a664-a1b82c95e306}{Foundation of CUDA-based GPU Programming}    
% \end{itemize}
% \begin{itemize}
%     % \vspace{0.5em}             
%     % 2022             
%     \item \href{https://github.com/user-attachments/assets/b4e013f1-d7b8-4505-a4f5-12e270097549}{PLL Design and Jitter Interpretation}                                  \hfill 2022
%     \item \href{https://github.com/user-attachments/assets/73933790-2b15-46ae-8ac2-c6eaf31d9e01}{Foundation of reinforcement learning}                                  
% \end{itemize}


% % KAIST
% \paragraph{\large{\href{https://gsds.kaist.ac.kr}{Korea Advanced Institute of Science and Technology (KAIST)}}}  \hfill Online, Korea
% \begin{itemize}
%     % \item \href{https://github.com/user-attachments/files/21983055/IMG_20250826_0002_NEW_0001.pdf}{AI in Biomedical Imaging - AI in Healthcare Technique}               \hfill 2025
%     \item \href{https://github.com/user-attachments/assets/b404e1df-2798-48de-8385-c9a6c7389b38}{Microdegree from Graduate School of Data Science}                      \hfill 2023 \\
% \end{itemize}



\section*{Leadership \& Volunteering}
\noindent
\large{\textbf{\href{https://innovationjapan.jst.go.jp}{Innovation Japan 2025}}}  \hfill Tokyo, Japan\\ % Company name and location
\textit{Student Staff}  \hfill Aug. 21 2025 | Aug. 22 2025  \\% Position and duration

\noindent
\large{\textbf{\href{https://www.iizuka.kyutech.ac.jp/admissions/open/open2025}{Open Campus 2025}} (Iizuka Campus, Kyushu Institute of Technology)}  \hfill Fukuoka, Japan\\ % Company name and location
\textit{Student Staff} \hfill Jul. 19 2025 | Jul. 20 2025 \\

\noindent
\large{\textbf{International Capstone Design Presentation with Partner Universities}} \hfill Fukuoka, Japan\\
\textit{Participant (Student Delegate)}  \hfill Jan. 2025 \\

\noindent
\large{\textbf{International Joint Research Meeting and Seminar}} \hfill Kumamoto, Japan \\
\textit{Participant (Student Delegate)} \hfill Aug. 9 2024 | Aug. 10 2024 \\


% \noindent
% \textbf{Hands}  \hfill Busan, Korea\\ % Company name and location
% Member \hfill Mar. 2021 | Feb. 2024 \\% Position and duration

\noindent
\large{\textbf{Embedded Systems Lab.}}  \hfill Busan, Korea\\ % Company name and location
\textit{Leader} \hfill Sep. 2023 | Dec. 2023% Position and duration
\begin{itemize}
    \item Embedded system design using ATMega128A
    \item Embedded system control using the I/O ports and potentiometer 
\end{itemize} 

\noindent
\large{\textbf{Donga Challenge}}  \hfill Busan, Korea\\ % Company name and location
\textit{Leader} \hfill Sep. 2023 | Dec. 2023% Position and duration
\begin{itemize}
    \item Self-directed learning and project-based teamwork initiative
    \item Teaching basic programming (Python) and data analysis using Pandas and NumPy
\end{itemize} 

\noindent
\large{\textbf{Digital System Lab.}}  \hfill Busan, Korea\\ % Company name and location
\textit{Leader} \hfill Sep. 2023 | Dec. 2023% Position and duration
\begin{itemize}
    \item Basic programming (C/C++) and embedded system design using ATMega128A
    \item Teaching digital logic design and Verilog HDL
\end{itemize} 

% \noindent
\large{\textbf{Dong-A Ping-Pong Association (DAPPA)}}  \hfill Busan, Korea\\ % Company name and location
\textit{President} \hfill Mar. 2021 | Feb. 2022 \\% Position and duration

\section*{Military Service}
\noindent
\textbf{\large{Republic of Korea Army}} \hfill Haman-gun, Gyeongsangnam-do, Korea \\
\textit{Sergeant, Active Duty Soldier} \hfill Apr. 2019 | Nov. 2020
\begin{itemize}
    % \item \href{https://github.com/user-attachments/files/21983055/IMG_20250826_0002_NEW_0001.pdf}{AI in Biomedical Imaging - AI in Healthcare Technique}               \hfill 2025
    \item \href{https://github.com/user-attachments/assets/584d8f84-5bbe-4a71-9ae3-bda5ce40bd6f}{Award for Outstanding Army Warrior}  
    \item \href{https://github.com/user-attachments/assets/d3b82b44-dff0-4d58-a944-747f46f8a50e}{Certificate of Appointment as Squad Leader}
    \item \href{https://github.com/user-attachments/assets/8565647d-3609-4866-8e18-fbdf9a9e227c}{Appointment Certificate as Squad Representative Soldier}
    \item \href{https://github.com/user-attachments/assets/f4f76912-92a7-440e-b8a2-4ce52b1ec39a}{Commendation for Exemplary Soldier}
    \item \href{https://github.com/user-attachments/assets/c5b564e8-6c7d-45dd-bae5-3c6d228cf3a5}{Award for Outstanding Army Warrior} \hfill 2020
\end{itemize}
% \begin{itemize}
% \end{itemize}




% Completed mandatory military service as a squad leader in an artillery unit; responsible for team coordination, safety checks, and equipment management.
% \begin{itemize}
%     \item \# Description of your job duty and responsibilities.
%     \item \# Description of your job duty and responsibilities.
%     \item \# Description of your job duty and responsibilities.
%     \item \# Description of your job duty and responsibilities.
% \end{itemize}


% \section*{ENGLISH \& GRE TESTS}
%     \begin{tabularx}{1\textwidth}{
%     >{\raggedright\arraybackslash}X 
%    >{\raggedright\arraybackslash}X }
%       \textbf{IELTS (Academic): 7.5} (overall score)&   \textbf{GRE General Test:}\\ \\
%     Listening: XX | Reading: XX & Quant: XXX | Verbal: XXX\\
%     Speaking: XX   | Writing: XX& Analytical writing: X\\
%     Test date: \#Month \#Year &Test date: \#Month \#Year
%     \end{tabularx}

% Skills Section


\section*{REFERENCES}
\textbf{Prof. Min-Chul Lee}\\
\textit{Associate Professor} \\
Department of Computer Science and Networks \hfill Iizuka, Fukuoka, Japan\\ 
Graduate School of Computer Science and Systems Engineering\\
Kyushu Institute of Technology
\begin{itemize}
    \item E-mail: \href{mailto:lee@csn.kyutech.ac.jp}{lee@csn.kyutech.ac.jp}
    \item Scholar Profiles: \href{https://orcid.org/0000-0001-8469-0288}{ORCiD} | \href{https://scholar.google.com/citations?user=9sv1-5QAAAAJ&hl=en&oi=ao}{Google Scholar}
    \item Homepage: \href{https://leelab.csn.kyutech.ac.jp}{3D Optical Imaging Systems Lab.}
\end{itemize}
\textbf{Prof. Myungjin Cho}\\
\textit{Professor} \\
School of ICT, Robotics and Mechanical Engineering, IITC \hfill Anseong, Kyonggi-do, Korea\\
Hankyong National University
\begin{itemize}
    \item E-mail: \href{mailto:mjcho@hknu.ac.kr}{mjcho@hknu.ac.kr}
    \item Scholar Profiles: \href{https://orcid.org/0000-0003-2896-770X}{ORCiD} | \href{https://scholar.google.com/citations?hl=en&user=WAvCU2kAAAAJ}{Google Scholar}
    \item Homepage: \href{https://3cholab.wordpress.com}{Computational, Holographic and Optical signal processing (CHO) Lab.}
\end{itemize}
\textbf{Prof. Bongsoon Kang}\\
\textit{Professor}
Department of Electronics Engineering \hfill Busan, Korea \\
College of Engineering \\
Dong-A University
\begin{itemize}
    \item E-mail: \href{mailto:bongsoon@dau.ac.kr}{bongsoon@dau.ac.kr}
    \item Scholar Profiles: \href{https://orcid.org/0000-0001-6716-5799}{ORCiD} | \href{https://www.dbpia.co.kr/author/authorDetail?ancId=810834}{DBpia}
    \item Homepage: \href{https://soc.donga.ac.kr/soc/Main.do}{SoC Design Lab.}
\end{itemize}
% End document
\end{document}
